%=====================================================
\begin{frame}{6.2. Граф ``Взаимные личные связи в организации''}

\begin{columns}
\begin{column}{0.7\textwidth} 
\centering
          \includegraphics[width=7cm, height=7cm]{graph7b.png}
\end{column}
\begin{column}{0.3\textwidth} 
\tiny

Количество взаимных связей~---~\valGBlinks.
\smallskip

Граф построен по ответам на вопросы С5 и С9.
\smallskip

\socioSizeComment
\smallskip

На диаграмме отображены только взаимные связи.
\smallskip

Размер узла пропорционален количеству связей.
\smallskip

Жёлтым цветом обозначен руководитель организации.
\smallskip

Изолированные узлы (т.~е. те, у которых нет взаимных личных связей) не показаны.


\end{column}
\end{columns}

\fontsize{6pt}{7}\selectfont
Именно взаимные связи создают устойчивые группы сотрудников. Посмотрите, образуются ли фигуры из сотрудников: 
пары (диады), треугольники (триады), цепочки, микрогруппы, где все участники взаимосвязаны, более сложные конфигурации.
Обратите внимание, номера каких сотрудников отсутствуют на рисунке. 


\end{frame}


