%=====================================================
\begin{frame}{4.2. Позиция педагогических работников по отношению горизонтального взаимодействия}

\tiny
Теперь посмотрим на позицию сотрудников по вопросу взаимодействия друг с другом. Как принято в коллективе, по мнению сотрудников, разрешать конфликты? По каким вопросам стоит обращаться к администрации, а по каким нет? Принято ли в принципе выражать своё мнение по отношению к поступкам коллег или это, по мнению сотрудников, дело администрации?
\bigskip

В данном разделе анализируются ответы на вопросы Б12, Б20, Б29.
\bigskip

\begin{itemize}

\item [Б12] Если у Вас возникла конфликтная ситуация с коллегой (например, по разделению обязанностей), как Вы ее решаете чаще всего?

\item [Б20] Если Вам покажется, что коллега-педагог (преподаватель, воспитатель) проявил несправедливость по отношению к обучающемуся (воспитаннику), что Вы сделаете?

\item [Б29] Выразите ли Вы свое отношение к коллеге, который/ая опоздал/а на проводимое им/ей занятие или мероприятие?

\end{itemize}

\end{frame}


