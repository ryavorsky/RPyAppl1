%=====================================================
\begin{frame}{5.1. Удовлетворенность работой (сводный результат) }


\tiny

На диаграмме представлен сводный результат по вопросам Б31, Б32.
\bigskip

\begin{columns}
\begin{column}{0.4\textwidth} 
\centering
\includegraphics[width=4cm, height=4cm]{pie71.png}
\end{column}
\begin{column}{0.6\textwidth} \begin{tabular}{l} 
 Ответили утвердительно   \\ 
(``да'' или ``скорее да, чем нет'')  ---   \valGAyesNumP\% \\ [0.3cm]
 Ответили отрицательно  \\ 
 (``нет'' или ``скорее нет, чем да'') ---  \valGAnoNumP\% \\ 
\end{tabular}
\end{column}
\end{columns}

На диаграмме представлены сводные результаты ответов на вопросы этого блока. Важными аспектами оценки в данном случае является: 
\begin {itemize}
\item Устраивает ли Вас уровень удовлетворенности работой сотрудников, если нет, что можно предпринять (если это необходимо) для его повышения?

\item Что в Вашей образовательной организации в наибольшей степени  влияет на уровень удовлетворенности работой?

\item Насколько высокий уровень удовлетворенности повышает эффективность деятельности организации?
\end {itemize}


\end{frame}


