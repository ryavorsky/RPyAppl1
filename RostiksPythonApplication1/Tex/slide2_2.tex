%=====================================================
\begin{frame}{2.2. Регламентированный и нерегламентированный обмен опытом}

\tiny
Фактически доверие в организации проявляется в обмене опытом и совместной деятельности. В данном разделе основное внимание мы уделяем взаимопосещениям учебных мероприятий. По данным международных исследований, лучшим способом профессионального развития (повышения квалификации) является наблюдение за работой коллеги и последующая рефлексия по этому поводу. Вопрос, который возникает в этом контексте, имеются ли в Вашей образовательной организации условия для реализации взаимопосещений. 
\bigskip

В данном разделе анализируются ответы на вопросы Б3, Б4, Б21, Б22, Б24.
\bigskip

\begin{itemize}

\item [Б3] Как часто за последний учебный год Вы посещали открытые уроки (занятия, мероприятия) педагогов (преподавателей)  Вашей образовательной организации?

\item [Б4] Как часто за последний учебный год Вы давали открытые уроки (занятия, мероприятия)?

\item [Б21] Как часто коллеги-педагоги (преподаватели, воспитатели) за последний учебный год посещали Ваши занятия и мероприятия (не открытые, т.е. которые вы специально не готовили)?

\item [Б22] Как часто за последний учебный год Вам приходилось бывать на занятиях и мероприятиях (не открытых, т.е. которые они специально не готовили) педагогов (преподавателей, воспитателей) Вашей образовательной организации?

\item [Б24] После посещения занятий  и мероприятий (открытых и неоткрытых)  в нашей образовательной организации  обычно:

\end{itemize}

\end{frame}


