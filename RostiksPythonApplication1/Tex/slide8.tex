%=====================================================
\begin{frame}{8. Заключение}

\tiny 
Вы получили первичные данные по проведенному исследованию. Эти результаты мы предоставляем только Вам и именно Вы должны решить, что и в каком объеме стоит обсуждать с коллегами. 
\smallskip

Следует принимать во внимание, что, как и всякое исследование, основанное на опросе, это отнюдь не претендует на то, чтобы поставить окончательный диагноз. Объективность полученных данных сильно зависит от условий, в которых они были получены (насколько тревожны были отвечающие, как они были проинструктированы и т.д.) Но если кто-то и осведомлен об этом, то это именно Вы.  Вы лучше всех знаете, можно ли этим данным доверять.
\smallskip

Но  если данные покажутся Вам значимыми, то они могут быть основанием для принятия решений, основанных не только на интуиции, но и на данных. Надеемся, что это исследование принесет пользу Вам и организации.


\end{frame}

%-----------------------------------------------------
\begin{frame}{}

\tiny 
Отчёт разрабатывали и готовили:

\begin{itemize}

\item Ушаков Константин Михайлович, д.п.н., профессор Института образования НИУ Высшая школа экономики, главный редактор журнала "Директор школы".

\item Фишбейн Дмитрий Ефимович, к.п.н., доцент Института образования НИУ Высшая школа экономики, главный редактор "Журнала руководителя управления образованием".

\item Яворский Ростислав Эдуардович, к.п.н., доцент Департамента анализа данных и искусственного интеллекта НИУ Высшая школа экономики.

\item Шиварев Павел Васильевич, заместитель главного редактора "Журнала руководителя управления образованием".

\item Кухарев Антон Иванович, магистр образования, заместитель директора МБОО СОШ № 25 г. Альметьевска Республики Татарстан

\end{itemize}

\end{frame}


