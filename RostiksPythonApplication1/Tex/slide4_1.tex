%=====================================================
\begin{frame}{4.1. Позиция администрации по отношению горизонтального взаимодействия}

\tiny

На диаграмме представлен результат по вопросу Б11.
\bigskip

\begin{itemize}
\item [Б11] Если у педагогов (преподавателей, воспитателей) возникла конфликтная ситуация (например, по разделению обязанностей), какие пути ее урегулирования они должны использовать прежде всего?
\end{itemize}

\begin{columns}
\begin{column}{0.4\textwidth} 
\centering
\includegraphics[width=4cm, height=4cm]{pie41.png}
\end{column}
\begin{column}{0.6\textwidth} \begin{tabular}{l} 
 Обратиться к руководству --- \valDAansA\ (\valDAansAp\%)  \\[0.5cm] 
Решить конфликтный вопрос самостоятельно,  \\
договорившись между собой ---   \valDAansB\ (\valDAansBp\%) \\[0.5cm]
Обсудить ситуацию с другими коллегами --- \valDAansC\ (\valDAansCp\%) \\[0.5cm]
Другое --- \valDAansD\ (\valDAansDp\%) \\[0.5cm]
\end{tabular}
\end{column}
\end{columns}

\end{frame}


