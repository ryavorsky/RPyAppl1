%=====================================================
\begin{frame}{4.2.б. Позиция педагогических работников по отношению горизонтального контроля}


\tiny

На диаграмме представлен результат по вопросу Б20:
\bigskip

\begin{itemize}
\item [Б20] Варианты реакции на несправедливость по отношению к обучающемуся, проявленную педагогом. ``Если Вам покажется, что коллега-педагог (преподаватель, воспитатель) проявил несправедливость по отношению к обучающемуся (воспитаннику), что Вы сделаете?''
\end{itemize}

\begin{columns}
\begin{column}{0.4\textwidth} 
\centering
\includegraphics[width=4cm, height=4cm]{pie42_b_.png}
\end{column}
\begin{column}{0.6\textwidth} \begin{tabular}{l} 
Утешите обучающегося (воспитанника) --- \valDBBansA\ (\valDBBansAp\%)  \\[0.5cm] 
Обсудите ситуацию с этим коллегой ---   \valDBBansB\ (\valDBBansBp\%) \\[0.5cm]
Обратитесь к администрации --- \valDBBansC\ (\valDBBansCp\%) \\[0.5cm]
Ничего, вдруг это мне просто показалось --- \valDBBansD\ (\valDBBansDp\%) \\[0.5cm]
Другое --- \valDBBansE\ (\valDBBansEp\%) \\[0.5cm]
\end{tabular}
\end{column}
\end{columns}

\end{frame}


