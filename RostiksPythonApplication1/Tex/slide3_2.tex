%=====================================================
\begin{frame}{3.2. Фактический уровень вертикального доверия}

\tiny

В этом разделе отчета  Вы найдете данные об аспектах профессионального доверия 
(в области образования и воспитания детей). Анализ данных этого раздела позволяет определить, 
какой уровень управления (директор, заместители, зав.отделениями) несёт основную нагрузку по 
профессиональному развитию педагогов и взаимодействию  с ними. 
Какой уровень наиболее значим для педагогов в этом отношении?
\bigskip

В данном разделе анализируются ответы на вопросы Б15, Б16, Б17.
\bigskip

\begin{itemize}

\item [Б15] Как часто Вы лично за последний учебный год по своей инициативе обращались (обсуждали, советовались) по вопросам преподавания и воспитания конкретных обучающихся (воспитанников) или классов (групп) к директору (заведующему)?

\item [Б16] Как часто Вы лично за последний учебный год по своей инициативе обращались (обсуждали, советовались) по вопросам преподавания и воспитания конкретных обучающихся (воспитанников) или классов (групп) к зам. директора (заведующего)?

\item [Б17] Как часто Вы лично за последний учебный год по своей инициативе обращались (советовались, обсуждали) по вопросам преподавания и воспитания конкретных обучающихся (воспитанников) или классов (групп)  к заведующим кафедрами  (методобъединений, отделений)?

\end{itemize}

\end{frame}


