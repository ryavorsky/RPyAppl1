%=====================================================
\begin{frame}{2.1. Декларируемый уровень доверия к коллегам }

\tiny
В этом разделе обсуждаем декларируемое доверие, то есть то, насколько Ваши коллеги в целом (можно сказать, теоретически) признают важность доверия, сотрудничества и обмена опытом. Это не означает, что они руководствуются этими принципами в своем повседневном профессиональном поведении, но считают правильным декларировать именно такую жизненную позицию.
\bigskip

В данном разделе анализируются ответы на вопросы Б13, Б14, Б10, Б25, Б30:
\bigskip

\begin{itemize}

\item [Б13] Есть ли у Вас профессиональные задачи, решение которых требует знакомства с опытом работы других педагогов (преподавателей, воспитателей) Вашей образовательной организации?

\item [Б14] Считаете ли Вы полезным и правильным посещение педагогами (преподавателями, воспитателями)  занятий и мероприятий (не открытых, т.е. специально не подготовленных), проводимых другими?

\item [Б10] С Вашей точки зрения, большинству коллег в Вашей образовательной организации можно доверять (доверие - уверенность в том, что, если Вы сказали коллеге о своих проблемах или ошибках, то эта информация не будет использована Вам во вред)?

\item[Б25] Как Вам кажется, нравится ли педагогам (преподавателям, воспитателям) то, что Вы посещаете их занятия и мероприятия?

\item[Б30] Спокойно ли Вам коллеги предоставляют свой кабинет (группу), оборудование для проведения урока (занятия или мероприятия)?

\end{itemize}

\end{frame}


