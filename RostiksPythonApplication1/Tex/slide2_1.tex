%=====================================================
\begin{frame}{2.1. Декларируемый уровень доверия к коллегам }

\tiny

В данном разделе анализируются ответы на вопросы Б13, Б14, Б10, Б25, Б30:
\bigskip

\begin{itemize}

\item [Б13] Наличие профессиональных задач, для решения которых требуется знакомство с опытом работы коллег: ``Есть ли у Вас профессиональные задачи, решение которых требует знакомства с опытом работы других педагогов (преподавателей, воспитателей) Вашей образовательной организации?''

\item [Б14] Полезность и правильность посещения педагогами  занятий и мероприятий, проводимых другими: ``Считаете ли Вы полезным и правильным посещение педагогами (преподавателями, воспитателями)  занятий и мероприятий (не открытых, т.е. специально не подготовленных), проводимых другими?''

\item [Б10] Возможность доверия большинству коллег в образовательной организации: ``С Вашей точки зрения, большинству коллег в Вашей образовательной организации можно доверять (доверие - уверенность в том, что, если Вы сказали коллеге о своих проблемах или ошибках, то эта информация не будет использована Вам во вред)?''

\item[Б25] Отношение педагогов к посещению их занятий респондентом: ``Как Вам кажется, нравится ли педагогам (преподавателям, воспитателям) то, что Вы посещаете их занятия и мероприятия?''

\item[Б30] Отношение коллег к предоставлению своего кабинета (группы)  для проведения урока (занятия или мероприятия)?: ``Спокойно ли Вам коллеги предоставляют свой кабинет (группу), оборудование для проведения урока (занятия или мероприятия)?''

\end{itemize}

\end{frame}


