%=====================================================
\begin{frame}{2.1.1. Признание ценности сотрудничества и доверия (сводный результат) }


\tiny

На диаграмме представлен сводный результат по вопросам Б13, Б14.
\bigskip

Б13. Есть ли у Вас профессиональные задачи, решение которых требует знакомства с опытом работы других педагогов (преподавателей, воспитателей) Вашей образовательной организации?
\smallskip

Б14. Считаете ли Вы полезным и правильным посещение педагогами (преподавателями, воспитателями)  занятий и мероприятий (не открытых, т.е. специально не подготовленных), проводимых другими?
\bigskip

\begin{columns}
\begin{column}{0.4\textwidth} 
\centering
\includegraphics[width=4cm, height=4cm]{pie211.png}
\end{column}
\begin{column}{0.6\textwidth} \begin{tabular}{l} 
 Ответили утвердительно   \\ 
(``да'' или ``скорее да, чем нет'')  ---   \valBAAyesNumP\% \\ [0.3cm]
 Ответили отрицательно  \\ 
 (``нет'' или ``скорее нет, чем да'') ---  \valBAAnoNumP\% \\ 
\end{tabular}
\end{column}
\end{columns}

На диаграмме показан сводный результат по вопросам о декларируемом доверии к коллегам. Насколько Вас устраивают полученные данные? Какой процент коллег, не признающих доверие как ценность, Вы бы считали критическим? Как бы Вы интерпретировали высокий процент людей в коллективе, кто ответил утвердительно на обозначенные вопросы?

\end{frame}


