%=====================================================
\begin{frame}{5.2. Удовлетворенность работой (по группам) }

\tiny

На диаграмме представлен результат по вопросам Б31, Б32 с разбивкой по возрасту.

\begin{tabular}{cccc}
\includegraphics[width=2.2cm, height=2.2cm]{pie72a.png} & 
\includegraphics[width=2.2cm, height=2.2cm]{pie72b.png} & 
\includegraphics[width=2.2cm, height=2.2cm]{pie72c.png} & 
\includegraphics[width=2.2cm, height=2.2cm]{pie72d.png} \\
до 25 лет &  25 -- 35  лет &  36 -- 55 лет & свыше 55 лет \\
\end{tabular}
\bigskip

То же с разбивкой по квалификационной категории.

\begin{tabular}{ccccc}
\includegraphics[width=2cm, height=2cm]{pie73a.png} & 
\includegraphics[width=2cm, height=2cm]{pie73b.png} & 
\includegraphics[width=2cm, height=2cm]{pie73c.png} & 
\includegraphics[width=2cm, height=2cm]{pie73d.png} & 
\includegraphics[width=2cm, height=2cm]{pie73e.png} \\
 Не проходили &  Аттестованы & 2-я &  1-я  & Высшая \\ 
  аттестацию   &  на соотв. & категория &  категория  & категория \\ 
\end{tabular}
\bigskip

На диаграмме - данные ответов на вопросы об удовлетворенности разных возрастных и категориальных групп. Однороден ли коллектив в своих позициях относительно удовлетворенности работой? Должен ли быть разным уровень удовлетворенности различных групп сотрудников? 
\end{frame}


