%=====================================================
\begin{frame}{3.1. Декларируемый уровень вертикального доверия  }

\tiny

Опять обсуждаем декларируемое доверие. На этот раз оцениваем, насколько сотрудники в целом доверяют руководству, считают, что о них заботятся. Видят ли они в руководстве «заботливых родителей» или равного коллегу? 
\smallskip

В этом контексте нам хочется, чтобы Вы поразмышляли о своей личной позиции: Вы желаете, чтобы сотрудники видели в Вас защитника и покровителя, или, наоборот, полагаете, что они должны уйти из «детской» позиции, стать самодостаточными, независимыми людьми и научиться преодолевать трудности самостоятельно? 

\end{frame}


