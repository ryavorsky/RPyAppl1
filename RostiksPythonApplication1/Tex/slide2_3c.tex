%=====================================================
\begin{frame}{2.3.3. Процедура работы команд}


\tiny

На диаграмме представлен результат по вопросу Б28.
\bigskip

\begin{itemize}
\item [Б28] Обсуждение вопросов в таких группах (командах) происходит, как правило:
\end{itemize}

\begin{columns}
\begin{column}{0.4\textwidth} 
\centering
\includegraphics[width=4cm, height=4cm]{pie233.png}
\end{column}
\begin{column}{0.6\textwidth} \begin{tabular}{l} 
 На перемене (в перерывах между занятиями) --- \valBCCansA\ (\valBCCansAp\%)  \\[0.5cm] 
Для этого есть специальное время и место ---   \valBCCansB\ (\valBCCansBp\%) \\[0.5cm]
Другое ---  \valBCCansC\ (\valBCCansCp\%) \\[0.5cm]
\end{tabular}
\end{column}
\end{columns}
\bigskip

Обратите внимание, где происходят встречи команд? Как Вы полагаете, какое место для встреч было бы оптимальным?
\end{frame}


