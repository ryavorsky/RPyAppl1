%=====================================================
\begin{frame}{3.2.2. Частотность обращения педагогов к администрации по вопросам преподавания и воспитания (по группам) }

\tiny

На диаграммах представлен результат по вопросам Б15, Б16, Б17 с разбивкой по возрасту.

\begin{tabular}{cccc}
\includegraphics[width=2.2cm, height=2.2cm]{pie322a.png} & 
\includegraphics[width=2.2cm, height=2.2cm]{pie322b.png} & 
\includegraphics[width=2.2cm, height=2.2cm]{pie322c.png} & 
\includegraphics[width=2.2cm, height=2.2cm]{pie322d.png} \\
до 25 лет &  25 -- 35  лет &  36 -- 55 лет & свыше 55 лет \\
\end{tabular}
\bigskip

То же с разбивкой по квалификационной категории.

\begin{tabular}{ccccc}
\includegraphics[width=2cm, height=2cm]{pie323a.png} & 
\includegraphics[width=2cm, height=2cm]{pie323b.png} & 
\includegraphics[width=2cm, height=2cm]{pie323c.png} & 
\includegraphics[width=2cm, height=2cm]{pie323d.png} & 
\includegraphics[width=2cm, height=2cm]{pie323e.png} \\
 Не проходили &  Аттестованы & 2-я &  1-я  & Высшая \\ 
  аттестацию   &  на соотв. & категория &  категория  & категория \\ 
\end{tabular}
\bigskip

На диаграммах данные ответов разных возрастных и категориальных групп. Как Вы полагаете: обращаться к администрации по профессиональным вопросам должны чаще молодые, а реже опытные? Какова Ваша позиция по группам разных квалификационных категорий? 

\end{frame}


