%=====================================================
\begin{frame}{7.1.2. Граф ``Взаимные актуальные профессиональные связи в организации''}

\begin{columns}
\begin{column}{0.7\textwidth} 
\centering
          \includegraphics[width=7cm, height=7cm]{graph8_1b.png}
\end{column}
\begin{column}{0.25\textwidth} 

\tiny
Граф построен по ответам на вопросы С6, С8.
\smallskip

На диаграмме отображены только взаимные связи.
\smallskip

Размер узла пропорционален количеству взаимных связей.
\smallskip

Изолированные узлы не показаны.
\smallskip

Жёлтым цветом обозначен руководитель организации.
\bigskip

Количество взаимных связей --- \valHABlinks.

\end{column}
\end{columns}

\fontsize{6pt}{7}\selectfont
Взаимные профессиональные связи создают устойчивые группы сотрудников. 
Какова природа этих связей, что их объединило?
Так же как  и по личным связям посмотрите, образуются ли фигуры из сотрудников: 
пары, цепочки, треугольники, более сложные конфигурации. 
Обратите внимание, номера каких сотрудников отсутствуют на рисунке. 


\end{frame}


