%=====================================================
\begin{frame}{2.1.2. Признание ценности сотрудничества и доверия (по группам) }

\tiny

На диаграммах представлен результат по вопросам Б13, Б14 с разбивкой по возрасту.

\begin{tabular}{cccc}
\includegraphics[width=2.2cm, height=2.2cm]{pie212a.png} & 
\includegraphics[width=2.2cm, height=2.2cm]{pie212b.png} & 
\includegraphics[width=2.2cm, height=2.2cm]{pie212c.png} & 
\includegraphics[width=2.2cm, height=2.2cm]{pie212d.png} \\
до 25 лет &  25 -- 35  лет &  36 -- 55 лет & свыше 55 лет \\
\end{tabular}
\bigskip

То же с разбивкой по квалификационной категории.

\begin{tabular}{ccccc}
\includegraphics[width=2cm, height=2cm]{pie213a.png} & 
\includegraphics[width=2cm, height=2cm]{pie213b.png} & 
\includegraphics[width=2cm, height=2cm]{pie213c.png} & 
\includegraphics[width=2cm, height=2cm]{pie213d.png} & 
\includegraphics[width=2cm, height=2cm]{pie213e.png} \\
 Не проходили &  Аттестованы & 2-я &  1-я  & Высшая \\ 
  аттестацию   &  на соотв. & категория &  категория  & категория \\ 
\end{tabular}
\bigskip

На диаграммах данные ответов разных возрастных и категориальных групп. 
Советуем отдельно обратить внимание на результаты педагогических работников высшей 
квалификационной категории – ведь это профессиональная элита, которая во многом определяет деятельность Вашей организации. 


\end{frame}

