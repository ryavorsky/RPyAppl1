% !TEX TS-program = pdflatex
% !TEX encoding = UTF-8 Unicode

\documentclass{beamer}

\usetheme{Boadilla}
\setbeamerfont{frametitle}{size=\small}

\usepackage[english,russian]{babel}
\usepackage[utf8]{inputenc}
\usepackage{times}

\newcommand{\docId}{Отчёт}
\newcommand{\theAuthor}{\textcopyright  АНО ``Образовательный квартал'' и журнал ``Директор школы''}
\newcommand{\myaa}{-1}
\newcommand{\myab}{-1}
\newcommand{\myac}{-1}
\newcommand{\myad}{-1}
\newcommand{\socioSizeCommentA}{В связи с малым количеством сотрудников в Вашей организации (меньше 7~человек) социометрический анализ не представляется возможным.}
\newcommand{\socioSizeCommentB}{В связи с количеством сотрудников в Вашей организации (не более 11~человек) анализировались только 2 верхних выбора по указанным вопросам.}
\newcommand{\socioSizeCommentC}{В связи с количеством сотрудников в Вашей организации (не более 16~человек) анализировались только 3 верхних выбора по указанным вопросам.}
\newcommand{\socioSizeCommentD}{В связи с количеством сотрудников в Вашей организации (не более 21~человек) анализировались только 4 верхних выбора по указанным вопросам.}
\newcommand{\socioSizeCommentE}{\ }


\makeatother
\setbeamertemplate{footline}
{
  \leavevmode%
  \hbox{%
  \begin{beamercolorbox}[wd=.2\paperwidth,ht=2.25ex,dp=1ex,center]{doc number}%
    \usebeamerfont{}\docId\  \No\  T-\internalId
  \end{beamercolorbox}%
  \begin{beamercolorbox}[wd=.7\paperwidth,ht=2.25ex,dp=1ex,center]{doc comment}%
    \usebeamerfont{}\theAuthor\hspace*{3em}
  \end{beamercolorbox}%
  \begin{beamercolorbox}[wd=.1\paperwidth,ht=2.25ex,dp=1ex,center]{doc comment}%
    \insertframenumber{} \hspace*{1ex}
  \end{beamercolorbox}}%
  \vskip0pt%
}
\makeatletter
\setbeamertemplate{navigation symbols}{}


\usenavigationsymbolstemplate{}
\subtitle{Анализ педагогического состава учебного заведения 
на основе проведенного анкетирования}

\author{}
\institute{}
\date{}

\begin{document}

\begin{frame}

\centering

\small{ \fullName\\  \vspace{.3cm}
Департамент образования Тюменской области  } \\ \vspace{2cm}

\large{\textcolor{blue}{АНАЛИЗ УРОВНЯ СОЦИАЛЬНОГО КАПИТАЛА}} \\ \vspace{.5cm}

\textcolor{blue}{Данные анкетирования административных и педагогических работников} \\ \vspace{2.5cm}

\small
Июнь -- август 2014 года

\end{frame}



%=====================================================
\begin{frame}{1.	Состав группы анкетируемых}

\tiny


\begin{itemize}
\item Количество административных и педагогических работников – \nTotal.
\item Количество заполнивших анкету – \nParticipated.
\item Административные работники – \numBoss.
\item Педагогические работники – \numTeacher.
\item Пол \\
\begin{tabular}{|c|c|} \hline
мужчины &  женщины \\ \hline
\numMen     &   \numWomen   \\ \hline
\end{tabular}

\item Возраст \\
\noindent
\begin{tabular}{|c|c|c|c|} \hline
до 25 лет &  25 -- 35  лет &  36 -- 55 лет & свыше 55 лет \\ \hline
\numYoung     &   \numMidAge         &   \numSenior        & \numOld  \\ \hline
\end{tabular}

\item Образование

\begin{tabular}{|c|c|c|c|c|c|c|} \hline
Среднее  & Среднее  & Высшее    & Иное  & Послевузовское \\
специальное & специальное & педагогическое & высшее & (аспирантура,\\
(педагог.)       & (не педагог.)   & (спец., маг., бак.) & &  докторантура) \\ \hline
\numEduA & \numEduB & \numEduC, \numEduD, \numEduE & \numEduF & \numEduG \\ \hline
\end{tabular}

\item Педагогический стаж \\
\noindent
\begin{tabular}{|c|c|c|c|} \hline
 до 5 лет &  6 -- 10 лет &  11 -- 20 лет & больше 20 лет \\ \hline
 \numExpA    &  \numExpB    & \numExpC & \numExpD \\ \hline
\end{tabular}

\item Стаж работы в данном образовательном учреждении \\
\noindent
\begin{tabular}{|c|c|c|c|} \hline
 до 5 лет &  6 -- 10  лет &  11 -- 20 лет & больше 20 лет \\ \hline
 \numExpHereA & \numExpHereB & \numExpHereC & \numExpHereD \\ \hline
\end{tabular}

\item Квалификационная категория \\
\noindent
\begin{tabular}{|c|c|c|c|c|} \hline
 Не  &  Аттестован & 2-я &  1-я  & Высшая \\ 
 сдавал &  на соотв. & категория &  категория  & категория \\ \hline
 \numTechCatA & \numTechCatB & \numTechCatC &  \numTechCatD &  \numTechCatE \\ \hline
\end{tabular}

\end{itemize}
\end{frame}



%=====================================================
\begin{frame}{1.1. Список сотрудников}
\begin{itemize}
\fontsize{3pt}{4}\selectfont
\input{nameslist.tex}
\end{itemize}
\end{frame}




%=====================================================
\begin{frame}{Раздел \No 4. Уровень сложности личных связей}

На графе отображены \textcolor{blue}{все} личные связи в организации.

\begin{columns}[T] 
\begin{column}{0.7\textwidth} 
\centering
          \includegraphics[width=6cm, height=6cm]{graph4.png}
\end{column}
\begin{column}{0.3\textwidth} % each column can also be its own environment
\tiny
Граф построен по ответам на два вопроса:
\begin{itemize}
\item К кому из коллег Вы обращаетесь за помощью или советом в сложной жизненной ситуации?
\item Кого из коллег Вы пригласите домой на свой день рождения?
\end{itemize}
\end{column}
\end{columns}
\end{frame}



%=====================================================
\begin{frame}{4.2. Граф взаимных личных связей}

\begin{columns}[T] 
\begin{column}{0.8\textwidth} 
\centering
          \includegraphics[width=9cm, height=8cm]{graph4b.png}
\end{column}
\begin{column}{0.2\textwidth} 
\tiny
Граф построен по ответам на два вопроса:
\begin{itemize}
\item К кому из коллег Вы обращаетесь за помощью или советом в сложной жизненной ситуации?
\item Кого из коллег Вы пригласите домой на свой день рождения?
\end{itemize}
\end{column}
\end{columns}
\end{frame}



%=====================================================
\begin{frame}{5.1.1 Актуальные профессиональные связи}

\begin{columns}[T] % contents are top vertically aligned
\begin{column}{0.8\textwidth} % alternative top-align that's better for graphics
\centering
          \includegraphics[width=9cm, height=8cm]{graph5_1a.png}
\end{column}
\begin{column}{0.2\textwidth} % each column can also be its own environment

Граф \textcolor{blue}{всех} профессиональных связей

\end{column}
\end{columns}
\end{frame}



%=====================================================
\begin{frame}{5.1.2 Актуальные профессиональные (взаимные) связи}

\begin{columns}[T] % contents are top vertically aligned
\begin{column}{0.8\textwidth} % alternative top-align that's better for graphics
\centering
          \includegraphics[width=9cm, height=8cm]{graph5_1b.png}
\end{column}
\begin{column}{0.2\textwidth} % each column can also be its own environment

Выделены \textcolor{blue}{взаимные} профессиональные связи

\end{column}
\end{columns}
\end{frame}



%=====================================================
\begin{frame}{Раздел 5. Сложность профессиональных связей}

{\large 5.2. Потенциальные профессиональные связи}

\begin{columns}[T] % contents are top vertically aligned
\begin{column}{0.7\textwidth} % alternative top-align that's better for graphics
\centering
          \includegraphics[width=6cm, height=6cm]{graph5_2a.png}
\end{column}
\begin{column}{0.3\textwidth} % each column can also be its own environment

\small
Граф \textcolor{blue}{всех} потенциальных профессиональных связей

Комментарии и пояснения к разделу 5.2

\end{column}
\end{columns}
\end{frame}



%=====================================================
\begin{frame}{Раздел 5. Сложность профессиональных связей}

{\large 5.2. Потенциальные профессиональные (взаимные) связи}

\begin{columns}[T] % contents are top vertically aligned
\begin{column}{0.7\textwidth} % alternative top-align that's better for graphics
\centering
          \includegraphics[width=6cm, height=6cm]{graph5_2b.png}
\end{column}
\begin{column}{0.3\textwidth} % each column can also be its own environment

\small
Выделены \textcolor{blue}{симметричные} профессиональн

Другие комментарии к разделу 5.2

\end{column}
\end{columns}
\end{frame}



%=====================================================
\begin{frame}{Раздел 5. Сложность профессиональных связей}
\begin{columns}[T] % contents are top vertically aligned
\begin{column}{0.7\textwidth} % alternative top-align that's better for graphics
\centering
          \includegraphics[width=6cm, height=6cm]{graph3.png}
\end{column}
\begin{column}{0.3\textwidth} % each column can also be its own environment
Другая важная мысль
\end{column}
\end{columns}
\end{frame}




\end{document}


