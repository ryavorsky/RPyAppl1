% !TEX TS-program = pdflatex
% !TEX encoding = UTF-8 Unicode

\documentclass{beamer}
\usetheme{Boadilla}

\usepackage[english,russian]{babel}
\usepackage[utf8]{inputenc}
\usepackage{times}

\newcommand{\docId}{Отчёт}
\newcommand{\theAuthor}{\textcopyright  АНО ``Образовательный квартал'' 2014 год}


\makeatother
\setbeamertemplate{footline}
{
  \leavevmode%
  \hbox{%
  \begin{beamercolorbox}[wd=.2\paperwidth,ht=2.25ex,dp=1ex,center]{doc number}%
    \usebeamerfont{}\insertshorttitle
  \end{beamercolorbox}%
  \begin{beamercolorbox}[wd=.7\paperwidth,ht=2.25ex,dp=1ex,center]{doc comment}%
    \usebeamerfont{}\theAuthor\hspace*{3em}
  \end{beamercolorbox}%
  \begin{beamercolorbox}[wd=.1\paperwidth,ht=2.25ex,dp=1ex,center]{doc comment}%
    \insertframenumber{} \hspace*{1ex}
  \end{beamercolorbox}}%
  \vskip0pt%
}
\makeatletter
\setbeamertemplate{navigation symbols}{}


\usenavigationsymbolstemplate{}
\title[МАОУ Заводопетровская СОШ] {Исследование социального капитала МАОУ Заводопетровская СОШ}

\subtitle{Анализ педагогического состава учебного заведения на основе проведенного анкетирования}


\subtitle{Анализ педагогического состава учебного заведения 
на основе проведенного анкетирования}

\author{}
\institute{}
\date{}

\begin{document}
\begin{frame}
  \titlepage
\end{frame}

\begin{frame}{Список сотрудников (номер, ф. и. о, должность)}

\begin{itemize}
\fontsize{3pt}{4}\selectfont
\input{names.tex}
\end{itemize}

\end{frame}

%=====================================================
\begin{frame}{4.1. Граф всех личных связей в организации}

\begin{columns}
\begin{column}{0.75\textwidth} 
\centering
          \includegraphics[width=8cm, height=8cm]{graph4a.png}
\end{column}
\begin{column}{0.25\textwidth}
\tiny
Граф \textcolor{blue}{всех} личных связей построен по ответам на два вопроса:
\smallskip

1. ``Если у Вас возникнет сложная/тяжелая жизненная ситуация и Вам нужна будет помощь, к кому из коллег Вы обратитесь?''
\smallskip

2. ``Кого из коллег Вы пригласите на свой день рождения к себе домой?''
\smallskip

Отображены \textcolor{blue}{все} связи. 
\smallskip

Размер узла пропорционален количеству входящих стрелок.

\end{column}
\end{columns}
\end{frame}



%=====================================================
\begin{frame}{Раздел \No 4. Уровень сложности личных связей}

Отображены \textcolor{blue}{симметричные} личные связи в организации.

\begin{columns}[T] % contents are top vertically aligned
\begin{column}{0.7\textwidth} 
\centering
          \includegraphics[width=6cm, height=6cm]{graph1.png}
\end{column}
\begin{column}{0.3\textwidth} 
\tiny
Граф построен по ответам на два вопроса:
\begin{itemize}
\item К кому из коллег Вы обращаетесь за помощью или советом в сложной жизненной ситуации?
\item Кого из коллег Вы пригласите домой на свой день рождения?
\end{itemize}
\end{column}
\end{columns}
\end{frame}



%=====================================================
\begin{frame}{5.1.1 Актуальные профессиональные связи}

\begin{columns} 
\begin{column}{0.75\textwidth} 
\centering
          \includegraphics[width=8cm, height=8cm]{graph5_1a.png}
\end{column}
\begin{column}{0.25\textwidth} 

\tiny
Граф \textcolor{blue}{всех} профессиональных связей построен по ответам на два вопроса:
\smallskip

1. ``Если у Вас возникают профессиональные проблемы, то с кем из коллег Вы советуетесь, к кому обращаетесь за помощью?''
\smallskip

2. ``Назовите тех коллег, к кому по разным причинам (помощь, знакомство с чужим опытом и пр.) Вы чаще ходите на занятия в настоящее время?''
\smallskip

Отображены \textcolor{blue}{все} связи. 
\smallskip

Размер узла пропорционален количеству входящих стрелок.

\end{column}
\end{columns}
\end{frame}



%=====================================================
\begin{frame}{5.1.2. Актуальные профессиональные связи (взаимные)}

\begin{columns}
\begin{column}{0.75\textwidth} 
\centering
          \includegraphics[width=8cm, height=8cm]{graph5_1b.png}
\end{column}
\begin{column}{0.25\textwidth} 

\tiny
Граф \textcolor{blue}{взаимных} профессиональных связей также построен по ответам на два вопроса:
\smallskip

- ``Если у Вас возникают профессиональные проблемы, то с кем из коллег Вы советуетесь, к кому обращаетесь за помощью?''
\smallskip

- ``Назовите тех коллег, к кому по разным причинам (помощь, знакомство с чужим опытом и пр.) Вы чаще ходите на занятия в настоящее время?''
\smallskip

На диаграмме отображены только \textcolor{blue}{взаимные} связи.
\smallskip

Размер узла пропорционален количеству связей.
\smallskip

Изолированные узлы не показаны.
\end{column}
\end{columns}
\end{frame}



%=====================================================
\begin{frame}{Раздел 5. Сложность профессиональных связей}

{\large 5.2. Потенциальные профессиональные связи}

\small
Граф \textcolor{blue}{всех} потенциальных профессиональных связей

\begin{columns}[T] % contents are top vertically aligned
\begin{column}{0.7\textwidth} % alternative top-align that's better for graphics
\centering
          \includegraphics[width=6cm, height=6cm]{graph4.png}
\end{column}
\begin{column}{0.3\textwidth} % each column can also be its own environment
Другая важная мысль
\end{column}
\end{columns}
\end{frame}



%=====================================================
\begin{frame}{Раздел 5.2.2. Потенциальные профессиональные связи}

\begin{columns} 
\begin{column}{0.75\textwidth}
\centering
          \includegraphics[width=8cm, height=8cm]{graph5_2b.png}
\end{column}
\begin{column}{0.25\textwidth} 

\tiny
Граф \textcolor{blue}{потенциальных взаимных} профессиональных связей также построен по ответам на два вопроса:
\smallskip

- ``Кого из коллег Вы хотели бы видеть в составе вновь организованной группы для решения какой-либо проблемы в области преподавания и воспитания?''
\smallskip

- ``Кого из коллег Вам было бы полезно видеть на своих уроках, занятиях и мероприятиях?''
\smallskip

На диаграмме отображены только \textcolor{blue}{взаимные} потенциальные профессиональные связи.
\smallskip

Размер узла пропорционален количеству связей.
\smallskip

Изолированные узлы не показаны.

\end{column}
\end{columns}
\end{frame}



%=====================================================
\begin{frame}{5.3. Удовлетворенность работой (группы по квалификационной категории) }

\tiny

На диаграммах представлен результат по вопросам Б31, Б32 с разбивкой по квалификационной категории.
\bigskip

\centering 

\begin{tabular}{|l|c|c|c|c|c|} \hline
  & Не проходил &  Аттестован & 2-я &  1-я  & Высшая \\ 
 &  аттестацию   &  на соотв. & категория &  категория  & категория \\ \hline
Ответили  & & & & & \\
утвердительно  & \valGCyesNumA  &  \valGCyesNumB  & \valGCyesNumC  & \valGCyesNumD  & \valGCyesNumE \\ 
(Да, cкорее да...) & & & & & \\ \hline
Ответили   & & & & & \\
отрицательно & \valGCnoNumA   & \valGCnoNumB  & \valGCnoNumC  & 
\valGCnoNumD & \valGCnoNumE \\ 
(Нет, cкорее нет...) & & & & & \\ \hline
\end{tabular}

\bigskip

\begin{tabular}{ccccc}
\includegraphics[width=2cm, height=2cm]{pie73a.png} & 
\includegraphics[width=2cm, height=2cm]{pie73b.png} & 
\includegraphics[width=2cm, height=2cm]{pie73c.png} & 
\includegraphics[width=2cm, height=2cm]{pie73d.png} & 
\includegraphics[width=2cm, height=2cm]{pie73e.png} \\
 Не проходили &  Аттестованы & 2-я &  1-я  & Высшая \\ 
  аттестацию   &  на соотв. & категория &  категория  & категория \\ 
\end{tabular}

\end{frame}




\end{document}


