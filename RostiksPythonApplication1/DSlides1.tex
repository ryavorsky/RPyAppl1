% !TEX TS-program = pdflatex
% !TEX encoding = UTF-8 Unicode


\documentclass{beamer}
\usetheme{Boadilla}
\usenavigationsymbolstemplate{}
\usepackage[english,russian]{babel}
\usepackage[utf8]{inputenc}
\usepackage{times}



\title[ГОУ СОШ \No 1234] % (optional, use only with long paper titles)
{Исследование социального капитала ГОУ Средняя Школа 1234 }

\subtitle
{Анализ педагогического состава учебного заведения на основе проведенного анкетирования}

\author[Июнь-август 2014]{Образовательный центр «Образовательный квартал»}

\date{\textcopyright АНО ДПО ОЦ «ОК» }


\begin{document}

\begin{frame}
  \titlepage
\end{frame}


\begin{frame}
\begin{block}{Анализ репутации}
На этой странице отображен граф связей, взаимодействий и доверия, построенный на основе 
тестирования, проведенного с помощью анкетирования.
\end{block}

\end{frame}

%=====================================================



\begin{frame}{Вводная социометрия}

На этой странице отображен граф связей, 

взаимодействий 
\begin{columns}[T] % contents are top vertically aligned
\begin{column}{6cm} % alternative top-align that's better for graphics
\centering
          \includegraphics[width=6cm, height=6cm]{graph1.png}
\end{column}
\begin{column}{0.3\textwidth} % each column can also be its own environment
     Contents of first column \\ split into two lines
\end{column}
\end{columns}


тестирования, проведенного с помощью анкетирования.

\end{frame}

%=====================================================

\begin{frame}{Социометрия 1}

На этой странице отображен граф связей, взаимодействий и доверия, построенный на основе 

\begin{columns}[T] % contents are top vertically aligned
\begin{column}{6cm} % alternative top-align that's better for graphics
\centering
          \includegraphics[width=6cm, height=6cm]{graph1a.png}
\end{column}
\begin{column}{0.3\textwidth} % each column can also be its own environment
     Тут должен быть поясняющий текст 

\tiny
\begin{itemize}
\item Про картинку
\item Про то, что на самом деле это всё значит
\item Как это можно использовать на практике
\item И много всего другого про то что На этой странице отображен граф связей, взаимодействий и доверия, построенный на основе

Выделены связи наиболее важного типа.
\end{itemize}
\end{column}
\end{columns}


Выделены связи наиболее важного типа.

\end{frame}

%=====================================================



\begin{frame}{Example of columns 2}

На этой странице отображен граф связей, 

взаимодействий 
\begin{columns}[T] % contents are top vertically aligned
\begin{column}{6cm} % alternative top-align that's better for graphics
\centering
          \includegraphics[width=6cm, height=6cm]{graph1b.png}
\end{column}
\begin{column}{0.3\textwidth} % each column can also be its own environment
     Contents of first column \\ split into two lines
\end{column}
\end{columns}


тестирования, проведенного с помощью анкетирования.

\end{frame}

\end{document}


